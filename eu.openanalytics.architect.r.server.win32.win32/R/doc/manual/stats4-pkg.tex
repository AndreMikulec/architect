
\chapter{The \texttt{stats4} package}
\HeaderA{stats4-package}{Statistical Functions using S4 Classes}{stats4.Rdash.package}
\aliasA{stats4}{stats4-package}{stats4}
\keyword{package}{stats4-package}
\keyword{models}{stats4-package}
%
\begin{Description}\relax
Statistical Functions using S4 classes.
\end{Description}
%
\begin{Details}\relax
This package contains functions and classes for statistics
using the \LinkA{S version 4}{methods.Rdash.package} class system.

The methods currently support maximum likelihood (function
\code{\LinkA{mle}{mle}()} returning class \code{"\LinkA{mle}{mle.Rdash.class}"}),
including methods for \code{\LinkA{logLik}{logLik.Rdash.methods}} for use
with \code{\LinkA{AIC}{AIC}}.
\end{Details}
%
\begin{Author}\relax
R Core Team and contributors worldwide

Maintainer: R Core Team \email{R-core@r-project.org}
\end{Author}
\HeaderA{coef-methods}{Methods for Function \code{coef} in Package \pkg{stats4}}{coef.Rdash.methods}
\aliasA{coef,ANY-method}{coef-methods}{coef,ANY.Rdash.method}
\aliasA{coef,mle-method}{coef-methods}{coef,mle.Rdash.method}
\aliasA{coef,summary.mle-method}{coef-methods}{coef,summary.mle.Rdash.method}
\keyword{methods}{coef-methods}
%
\begin{Description}\relax
Extract the coefficient vector from \code{"mle"} objects.
\end{Description}
%
\begin{Section}{Methods}
\begin{description}

\item[\code{signature(object = "ANY")}] Generic function: see
\code{\LinkA{coef}{coef}}.
\item[\code{signature(object = "mle")}] Extract the full coefficient
vector (including any fixed coefficients) from the fit.
\item[\code{signature(object = "summary.mle")}] Extract the
coefficient vector and standard errors from the summary of the
fit.

\end{description}

\end{Section}
\HeaderA{confint-methods}{Methods for Function \code{confint} in Package \pkg{stats4}}{confint.Rdash.methods}
\aliasA{confint,ANY-method}{confint-methods}{confint,ANY.Rdash.method}
\aliasA{confint,mle-method}{confint-methods}{confint,mle.Rdash.method}
\aliasA{confint,profile.mle-method}{confint-methods}{confint,profile.mle.Rdash.method}
\keyword{methods}{confint-methods}
%
\begin{Description}\relax
Generate confidence intervals
\end{Description}
%
\begin{Section}{Methods}
\begin{description}

\item[\code{signature(object = "ANY")}] Generic function: see
\code{\LinkA{confint}{confint}}.
\item[\code{signature(object = "mle")}] First generate profile and
then confidence intervals from the profile.
\item[\code{signature(object = "profile.mle")}] Generate confidence
intervals based on likelihood profile.

\end{description}

\end{Section}
\HeaderA{logLik-methods}{Methods for Function \code{logLik} in Package \pkg{stats4}}{logLik.Rdash.methods}
\aliasA{logLik,ANY-method}{logLik-methods}{logLik,ANY.Rdash.method}
\aliasA{logLik,mle-method}{logLik-methods}{logLik,mle.Rdash.method}
\keyword{methods}{logLik-methods}
%
\begin{Description}\relax
Extract the maximized log-likelihood from \code{"mle"} objects.
\end{Description}
%
\begin{Section}{Methods}
\begin{description}

\item[\code{signature(object = "ANY")}] Generic function: see
\code{\LinkA{logLik}{logLik}}.
\item[\code{signature(object = "mle")}] Extract log-likelihood from
the fit.

\end{description}

\end{Section}
%
\begin{Note}\relax
The \code{mle} method does not know about the number of observations
unless \code{nobs} was specified on the call and so may not be
suitable for use with \code{\LinkA{BIC}{BIC}}.
\end{Note}
\HeaderA{mle}{Maximum Likelihood Estimation}{mle}
\keyword{models}{mle}
%
\begin{Description}\relax
Estimate parameters by the method of maximum likelihood.
\end{Description}
%
\begin{Usage}
\begin{verbatim}
mle(minuslogl, start = formals(minuslogl), method = "BFGS",
    fixed = list(), nobs, ...)
\end{verbatim}
\end{Usage}
%
\begin{Arguments}
\begin{ldescription}
\item[\code{minuslogl}] Function to calculate negative log-likelihood.
\item[\code{start}] Named list. Initial values for optimizer.
\item[\code{method}] Optimization method to use. See \code{\LinkA{optim}{optim}}.
\item[\code{fixed}] Named list.  Parameter values to keep fixed during
optimization.
\item[\code{nobs}] optional integer: the number of observations, to be used for
e.g. computing \code{\LinkA{BIC}{BIC}}.
\item[\code{...}] Further arguments to pass to \code{\LinkA{optim}{optim}}.
\end{ldescription}
\end{Arguments}
%
\begin{Details}\relax
The \code{\LinkA{optim}{optim}} optimizer is used to find the minimum of the
negative log-likelihood.  An approximate covariance matrix for the
parameters is obtained by inverting the Hessian matrix at the optimum.
\end{Details}
%
\begin{Value}
An object of class \code{\LinkA{mle-class}{mle.Rdash.class}}.
\end{Value}
%
\begin{Note}\relax
Be careful to note that the argument is -log L (not -2 log L). It
is for the user to ensure that the likelihood is correct, and that
asymptotic likelihood inference is valid.
\end{Note}
%
\begin{SeeAlso}\relax
\code{\LinkA{mle-class}{mle.Rdash.class}}
\end{SeeAlso}
%
\begin{Examples}
\begin{ExampleCode}
## Avoid printing to unwarranted accuracy
od <- options(digits = 5)
x <- 0:10
y <- c(26, 17, 13, 12, 20, 5, 9, 8, 5, 4, 8)

## Easy one-dimensional MLE:
nLL <- function(lambda) -sum(stats::dpois(y, lambda, log=TRUE))
fit0 <- mle(nLL, start = list(lambda = 5), nobs = NROW(y))
# For 1D, this is preferable:
fit1 <- mle(nLL, start = list(lambda = 5), nobs = NROW(y),
            method = "Brent", lower = 1, upper = 20)
stopifnot(nobs(fit0) == length(y))

## This needs a constrained parameter space: most methods will accept NA
ll <- function(ymax = 15, xhalf = 6) {
    if(ymax > 0 && xhalf > 0)
      -sum(stats::dpois(y, lambda = ymax/(1+x/xhalf), log = TRUE))
    else NA
}
(fit <- mle(ll, nobs = length(y)))
mle(ll, fixed = list(xhalf = 6))
## alternative using bounds on optimization
ll2 <- function(ymax = 15, xhalf = 6)
    -sum(stats::dpois(y, lambda=ymax/(1+x/xhalf), log = TRUE))
mle(ll2, method = "L-BFGS-B", lower = rep(0, 2))

AIC(fit)
BIC(fit)

summary(fit)
logLik(fit)
vcov(fit)
plot(profile(fit), absVal = FALSE)
confint(fit)

## Use bounded optimization
## The lower bounds are really > 0,
## but we use >=0 to stress-test profiling
(fit2 <- mle(ll, method = "L-BFGS-B", lower = c(0, 0)))
plot(profile(fit2), absVal=FALSE)

## a better parametrization:
ll3 <- function(lymax = log(15), lxhalf = log(6))
    -sum(stats::dpois(y, lambda=exp(lymax)/(1+x/exp(lxhalf)), log=TRUE))
(fit3 <- mle(ll3))
plot(profile(fit3), absVal = FALSE)
exp(confint(fit3))

options(od)
\end{ExampleCode}
\end{Examples}
\HeaderA{mle-class}{Class \code{"mle"} for Results of Maximum Likelihood Estimation}{mle.Rdash.class}
\aliasA{nobs,mle-method}{mle-class}{nobs,mle.Rdash.method}
\keyword{classes}{mle-class}
%
\begin{Description}\relax
This class encapsulates results of a generic maximum
likelihood procedure.
\end{Description}
%
\begin{Section}{Objects from the Class}
Objects can be created by calls of the form \code{new("mle", ...)}, but
most often as the result of a call to \code{\LinkA{mle}{mle}}.
\end{Section}
%
\begin{Section}{Slots}
\begin{description}

\item[\code{call}:] Object of class \code{"language"}.
The call to \code{\LinkA{mle}{mle}}.
\item[\code{coef}:] Object of class \code{"numeric"}.  Estimated
parameters.
\item[\code{fullcoef}:] Object of class \code{"numeric"}.
Fixed and estimated parameters.
\item[\code{vcov}:] Object of class \code{"matrix"}.  Approximate
variance-covariance matrix.
\item[\code{min}:] Object of class \code{"numeric"}.  Minimum value
of objective function.
\item[\code{details}:] a \code{"\LinkA{list}{list}"}, as returned from
\code{\LinkA{optim}{optim}}.
\item[\code{minuslogl}:] Object of class \code{"function"}.  The
negative loglikelihood function.
\item[\code{nobs}:] \code{"\LinkA{integer}{integer}"} of length one.  The
number of observations (often \code{NA}, when not set in call
explicitly).
\item[\code{method}:] Object of class \code{"character"}.  The
optimization method used.

\end{description}

\end{Section}
%
\begin{Section}{Methods}
\begin{description}

\item[confint] \code{signature(object = "mle")}: Confidence
intervals from likelihood profiles.
\item[logLik] \code{signature(object = "mle")}: Extract maximized
log-likelihood.
\item[profile] \code{signature(fitted = "mle")}: Likelihood profile
generation.
\item[nobs] \code{signature(object = "mle")}: Number of
observations, here simply accessing the \code{nobs} slot mentioned above.
\item[show] \code{signature(object = "mle")}: Display object briefly.
\item[summary] \code{signature(object = "mle")}: Generate object summary.
\item[update] \code{signature(object = "mle")}:  Update fit.
\item[vcov] \code{signature(object = "mle")}: Extract
variance-covariance matrix.

\end{description}

\end{Section}
\HeaderA{plot-methods}{Methods for Function \code{plot} in Package \pkg{stats4}}{plot.Rdash.methods}
\aliasA{plot,ANY,ANY-method}{plot-methods}{plot,ANY,ANY.Rdash.method}
\aliasA{plot,profile.mle,missing-method}{plot-methods}{plot,profile.mle,missing.Rdash.method}
\keyword{methods}{plot-methods}
%
\begin{Description}\relax
Plot profile likelihoods for \code{"mle"} objects.
\end{Description}
%
\begin{Usage}
\begin{verbatim}
## S4 method for signature 'profile.mle,missing'
plot(x, levels, conf = c(99, 95, 90, 80, 50)/100, nseg = 50,
     absVal = TRUE, ...)
\end{verbatim}
\end{Usage}
%
\begin{Arguments}
\begin{ldescription}
\item[\code{x}] an object of class \code{"profile.mle"} 
\item[\code{levels}] levels, on the scale of the absolute value of a t
statistic, at which to interpolate intervals.  Usually \code{conf}
is used instead of giving \code{levels} explicitly.
\item[\code{conf}] a numeric vector of confidence levels for profile-based
confidence intervals on the parameters.
\item[\code{nseg}] an integer value giving the number of segments to use in
the spline interpolation of the profile t curves.
\item[\code{absVal}] a logical value indicating whether or not the plots
should be on the scale of the absolute value of the profile t.
Defaults to \code{TRUE}.
\item[\code{...}] other arguments to the \code{plot} function can be passed here.
\end{ldescription}
\end{Arguments}
%
\begin{Section}{Methods}
\begin{description}

\item[\code{signature(x = "ANY", y = "ANY")}] Generic function: see
\code{\LinkA{plot}{plot}}.
\item[\code{signature(x = "profile.mle", y = "missing")}] Plot
likelihood profiles for \code{x}.

\end{description}

\end{Section}
\HeaderA{profile-methods}{Methods for Function \code{profile} in Package \pkg{stats4}}{profile.Rdash.methods}
\aliasA{profile,ANY-method}{profile-methods}{profile,ANY.Rdash.method}
\aliasA{profile,mle-method}{profile-methods}{profile,mle.Rdash.method}
\keyword{methods}{profile-methods}
%
\begin{Description}\relax
Profile likelihood for \code{"mle"} objects.
\end{Description}
%
\begin{Usage}
\begin{verbatim}
## S4 method for signature 'mle'
profile(fitted, which = 1:p, maxsteps = 100, alpha = 0.01,
        zmax = sqrt(qchisq(1 - alpha, 1L)), del = zmax/5,
        trace = FALSE, ...)
\end{verbatim}
\end{Usage}
%
\begin{Arguments}
\begin{ldescription}
\item[\code{fitted}] Object to be profiled
\item[\code{which}] Optionally select subset of parameters to profile.
\item[\code{maxsteps}] Maximum number of steps to bracket \code{zmax}.
\item[\code{alpha}] Significance level corresponding to \code{zmax}, based on
a Scheffe-style multiple testing interval.  Ignored
if \code{zmax} is specified.
\item[\code{zmax}] Cutoff for the profiled value of the signed root-likelihood.
\item[\code{del}] Initial stepsize on root-likelihood scale.
\item[\code{trace}] Logical. Print intermediate results.
\item[\code{...}] Currently unused.
\end{ldescription}
\end{Arguments}
%
\begin{Details}\relax
The profiling algorithm tries to find an approximately evenly spaced
set of at least five parameter values (in each direction from the
optimum) to cover  
the root-likelihood function. Some care is taken to try and get sensible
results in cases of high parameter curvature. Notice that it may not
always be possible to obtain the cutoff value, since the likelihood
might level off.
\end{Details}
%
\begin{Value}
An object of class \code{"profile.mle"}, see
\code{"profile.mle-class"}.
\end{Value}
%
\begin{Section}{Methods}
\begin{description}

\item[\code{signature(fitted = "ANY")}] Generic function: see
\code{\LinkA{profile}{profile}}.
\item[\code{signature(fitted = "mle")}] Profile the likelihood in
the vicinity of the optimum of an \code{"mle"} object.

\end{description}

\end{Section}
\HeaderA{profile.mle-class}{Class \code{"profile.mle"}; Profiling information for \code{"mle"} object}{profile.mle.Rdash.class}
\keyword{classes}{profile.mle-class}
%
\begin{Description}\relax
Likelihood profiles along each parameter of likelihood function
\end{Description}
%
\begin{Section}{Objects from the Class}
Objects can be created by calls of the form \code{new("profile.mle",
  ...)}, but most often by invoking \code{profile} on an "mle" object.
\end{Section}
%
\begin{Section}{Slots}
\begin{description}

\item[\code{profile}:] Object of class \code{"list"}. List of
profiles, one for each requested parameter. Each profile is a data
frame with the first column called \code{z} being the signed square
root of the -2 log likelihood ratio, and the others being the
parameters with names prefixed by \code{par.vals.}
\item[\code{summary}:] Object of class \code{"summary.mle"}. Summary
of object being profiled.

\end{description}

\end{Section}
%
\begin{Section}{Methods}
\begin{description}

\item[confint] \code{signature(object = "profile.mle")}: Use profile
to generate approximate confidence intervals for parameters.
\item[plot] \code{signature(x = "profile.mle", y = "missing")}: Plot
profiles for each parameter.

\end{description}

\end{Section}
%
\begin{SeeAlso}\relax
\code{\LinkA{mle}{mle}}, \code{\LinkA{mle-class}{mle.Rdash.class}}, \code{\LinkA{summary.mle-class}{summary.mle.Rdash.class}} 
\end{SeeAlso}
\HeaderA{show-methods}{Methods for Function \code{show} in Package \pkg{stats4}}{show.Rdash.methods}
\aliasA{show,mle-method}{show-methods}{show,mle.Rdash.method}
\aliasA{show,summary.mle-method}{show-methods}{show,summary.mle.Rdash.method}
\keyword{methods}{show-methods}
%
\begin{Description}\relax
Show objects of classes \code{mle} and \code{summary.mle}
\end{Description}
%
\begin{Section}{Methods}
\begin{description}

\item[\code{signature(object = "mle")}] Print simple summary of
\code{mle} object.  Just the coefficients and the call.
\item[\code{signature(object = "summary.mle")}] Shows call, table of
coefficients and standard errors, and \eqn{-2 \log L}{}.

\end{description}

\end{Section}
\HeaderA{summary-methods}{Methods for Function \code{summary} in Package \pkg{stats4}}{summary.Rdash.methods}
\aliasA{summary,ANY-method}{summary-methods}{summary,ANY.Rdash.method}
\aliasA{summary,mle-method}{summary-methods}{summary,mle.Rdash.method}
\keyword{methods}{summary-methods}
%
\begin{Description}\relax
Summarize objects
\end{Description}
%
\begin{Section}{Methods}
\begin{description}

\item[\code{signature(object = "ANY")}] Generic function
\item[\code{signature(object = "mle")}] Generate a summary as an
object of class \code{"summary.mle"}, containing estimates,
asymptotic SE, and value of \eqn{-2 \log L}{}.

\end{description}

\end{Section}
\HeaderA{summary.mle-class}{Class \code{"summary.mle"}, Summary of \code{"mle"} Objects}{summary.mle.Rdash.class}
\keyword{classes}{summary.mle-class}
%
\begin{Description}\relax
Extract of "mle" object
\end{Description}
%
\begin{Section}{Objects from the Class}
Objects can be created by calls of the form \code{new("summary.mle",
  ...)}, but most often by invoking \code{summary} on an "mle" object.
They contain values meant for printing by \code{show}.
\end{Section}
%
\begin{Section}{Slots}
\begin{description}

\item[\code{call}:] Object of class \code{"language"} The call that
generated the "mle" object.
\item[\code{coef}:] Object of class \code{"matrix"}. Estimated
coefficients and standard errors 
\item[\code{m2logL}:] Object of class \code{"numeric"}. Minus twice
the log likelihood.

\end{description}

\end{Section}
%
\begin{Section}{Methods}
\begin{description}

\item[show] \code{signature(object = "summary.mle")}: Pretty-prints
\code{object}  
\item[coef] \code{signature(object = "summary.mle")}: Extracts the
contents of the \code{coef} slot

\end{description}

\end{Section}
%
\begin{SeeAlso}\relax
\code{\LinkA{summary}{summary}}, \code{\LinkA{mle}{mle}}, \code{\LinkA{mle-class}{mle.Rdash.class}} 
\end{SeeAlso}
\HeaderA{update-methods}{Methods for Function \code{update} in Package \pkg{stats4}}{update.Rdash.methods}
\aliasA{update,ANY-method}{update-methods}{update,ANY.Rdash.method}
\aliasA{update,mle-method}{update-methods}{update,mle.Rdash.method}
\keyword{methods}{update-methods}
%
\begin{Description}\relax
Update \code{"mle"} objects.
\end{Description}
%
\begin{Usage}
\begin{verbatim}
## S4 method for signature 'mle'
update(object, ..., evaluate = TRUE)
\end{verbatim}
\end{Usage}
%
\begin{Arguments}
\begin{ldescription}
\item[\code{object}] An existing fit.
\item[\code{...}] Additional arguments to the call, or arguments with
changed values. Use \code{name=NULL} to remove the argument \code{name}.
\item[\code{evaluate}] If true evaluate the new call else return the call.
\end{ldescription}
\end{Arguments}
%
\begin{Section}{Methods}
\begin{description}

\item[\code{signature(object = "ANY")}] Generic function: see
\code{\LinkA{update}{update}}.
\item[\code{signature(object = "mle")}] Update a fit.

\end{description}

\end{Section}
%
\begin{Examples}
\begin{ExampleCode}
x <- 0:10
y <- c(26, 17, 13, 12, 20, 5, 9, 8, 5, 4, 8)
ll <- function(ymax=15, xhalf=6)
    -sum(stats::dpois(y, lambda=ymax/(1+x/xhalf), log=TRUE))
fit <- mle(ll)
## note the recorded call contains ..1, a problem with S4 dispatch
update(fit, fixed=list(xhalf=3))
\end{ExampleCode}
\end{Examples}
\HeaderA{vcov-methods}{Methods for Function \code{vcov} in Package \pkg{stats4}}{vcov.Rdash.methods}
\aliasA{vcov,ANY-method}{vcov-methods}{vcov,ANY.Rdash.method}
\aliasA{vcov,mle-method}{vcov-methods}{vcov,mle.Rdash.method}
\keyword{methods}{vcov-methods}
%
\begin{Description}\relax
Extract the approximate variance-covariance matrix from \code{"mle"}
objects.
\end{Description}
%
\begin{Section}{Methods}
\begin{description}

\item[\code{signature(object = "ANY")}] Generic function: see
\code{\LinkA{vcov}{vcov}}.
\item[\code{signature(object = "mle")}] Extract the estimated
variance-covariance matrix for the estimated parameters (if any).

\end{description}

\end{Section}
\clearpage
